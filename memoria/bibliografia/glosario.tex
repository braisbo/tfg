%%%%%%%%%%%%%%%%%%%%%%%%%%%%%%%%%%%%%%%%%%%%%%%%%%%%%%%%%%%%%%%%%%%%%%%%%%%%%%%%
% Obxectivo: Lista de termos empregados no documento,                          %
%            xunto cos seus respectivos significados.                          %
%%%%%%%%%%%%%%%%%%%%%%%%%%%%%%%%%%%%%%%%%%%%%%%%%%%%%%%%%%%%%%%%%%%%%%%%%%%%%%%%

\newglossaryentry{bytecode}{
  name=bytecode,
  description={Código independente da máquina que xeran compiladores de determinadas linguaxes (Java, Erlang,\dots) e que é executado polo correspondente intérprete.}
}
\newglossaryentry{TC}{
  name=TC,
  description={Tomografía Computerizada.}
}
\newglossaryentry{vc}{
    name=Virtuality Continuum,
    description={Representación completa del espectro de posibilidades tecnológicas entre un mundo completamente real  y uno completamente virtual}
}
\newglossaryentry{pinhole}{
    name=Pinhole camera model,
    description={Descripcion de la relación matemática entre las coordenadas de un punto en el espacio de tres dimensiones y su proyección en el plano de imagen de una supuesta cámara estenopeica ideal}
}